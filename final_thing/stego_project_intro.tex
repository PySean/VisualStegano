%%stego_project_intro, by Sean Soderman, Matthew Seaworth, Dylan Soderman%%

\documentclass[11pt]{article}
\usepackage{times}
\usepackage{anysize}
\marginsize{1 in}{1 in}{1 in}{1 in}

\begin{document}

\title{Spread Spectrum Steganography}
\author{Matthew Seaworth, Sean Soderman, Dylan Soderman}
\date{\today}
\maketitle

\section{Project Description}
We are implementing a spread spectrum image steganography algorithm with three main components. The spread spectrum algorithm is slightly
modified using a uniform distribution that is then spaced to increase the distance between bitvalues to the maximum possible distance between
the two. Then afterwards it is transformed into a normal distribution and added to the cover image. On the other end, the images is processed
to extract the original cover image from the steganography image. The extraction of the cover image is not exact and introduces bit errors.
The noise is then duplicated on the other end with an identical seed and the closest stream is picked to extract the original message. At both
ends error correcting is used before and after the encoding and decoding to help reduce errors.


\section{General Approach}
\indent \indent The Spread Spectrum process contains two phases: encoding and decoding. Encoding is moslty straight forward. Starting with
a shared key each bit has a corresponding psuedo-random long double generated on the domain from zero to one. If the bit is zero the normal
stream is used, but if the bit is one the prime stream is used. Prime in this usage just means alternate. The prime stream is half of the range
away from the generated bit but also within the domain. Meaning, if the default stream is below the middle point half the range is added whereas if
it is above the middle point half the range is subtracted. These values are then fed into and inverse gaussian cumulative distribution function.
This function is unbounded at both zero and one so the entire range is translated into a new domain clipped enough to allow reasonable values of the
function through. Once it has gone through this function the stream is now in the form of a random value from a normal distribution. The Spread Spectrum
decoding process is just a reverse of the encoding using a recovered cover image and the stego image to guess at the value of the bit at that location.
The recovered cover image is generated by using alpha-trimmed mean filter to correct the noise data inserted into the image. Additionally the message undergoes
convolutional code to account for the imperfect nature of the alpha-trimmed mean filter combined with the decoding process around extreme values. The
convolutional code once extracted undergoes a decoding using Viterbi decoding which produces the original image.  \\

\indent
The program takes inputs of a picture a message and a seed for encoding and a picture an output file and a seed for decoding. The picture given to the
decoding function should be the picture produced from the encoding function. The decoding function does not require the original picture to operate as
long as the error correcting code is sufficient to deal with the errors produced by the alpha-trimmed mean filter and decoding.

\newpage
\begin{thebibliography}{9}

\bibitem{MarvelLM}
Marvel, L. M., C. R. Boncelet, and C. T. Retter. 
\textit{Spread Spectrum Image Steganography.}
\textit{\underline{IEEE transactions on image processing : a publication of the IEEE Signal Processing Society}}
8.8 (1999): 1075-83.

\bibitem{Normal Code Source}
John Burkardt
\textit{NORMAL Normal Random Number Generators }
\textit{\underline{http://people.sc.fsu.edu/~jburkardt/c\_ src/normal/normal.html}}
14 August 2004

\bibitem{Inverse CDF Source}
Natarajan V., (Kanchipuram (near Madras), India)
\textit{Normal Inverse CDF}
\textit{\underline{http://home.online.no/~pjacklam/notes/invnorm/impl/natarajan/normsinv.h}}
\end{thebibliography}



\end{document}
