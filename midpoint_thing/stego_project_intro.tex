%%stego_project_intro, by Sean Soderman, Matthew Seaworth%

\documentclass{article}
\usepackage{times}


\begin{document}

\title{Grayscale Digital Stego w/ Modified Arnold Transformations}
\author{Sean Soderman}
\date{\today}
\maketitle

\section{Project Description}
We are using a modified Arnold's Transform(AT) to permute the spatial coordinants of the bitplanes in the message images
before embedding in order to increase entropy of the least significant bits in the coresponding steganography images. The transform
matrix is changed to include the values i and i plus one as the entries in the first row. Then several of these matrixes are applied
to the image for a number of times in a particular order known only by the sender and reciever. This requires a transfer of private
keys beforehand to user the correct AT matrices and the correct number of rotations. The period of the AT of an arbitrarily sized
matrix can be given by the least common multiple of the period of the independent factors of the side length (N). A set of numbers
are independent factors of another number if their multiple equals that number and they share a greatest commmon divisor of one.


\section{General Approach}

\begin{thebibliography}{9}
\bibitem{mishra12}
Minati Mishra, Ashanta Ranjan Routray, and Sunit Kumar:
\textit{"High Security Image Steganography with Modified Arnold's Cat Map"}
\textit{\underline{International Journal of Computer Applications}}, pp. 16-20,
Vol. 37, January 2012.

\bibitem{li05}
Li Bing, Xu Jia-wei:
\textit{Period of Arnold transformation and its application in image scrambling}
\textit{\underline{Journal of South Central University of Technology}}, pp. 278-282,
Vol. 12, October 2005.

\end{thebibliography}


\end{document}
